\documentclass[11pt, draft]{beamer}
\usepackage[utf8]{inputenc}
\usepackage[T1]{fontenc}
\usepackage[francais]{babel}
\usepackage{lmodern}
\usepackage{amsmath, amssymb, graphicx, amsthm, hyperref}
%\usepackage{fancyhdr}
%\usepackage{pslatex} % À enlever si problème, soit disant pour rendre le document plus "lisible" 

\usetheme{Madrid}

\title{La transformée de \textsc{Gelfand}}
\subtitle{Lien avec la physique, et théorie des algèbres de Banach}
\author{Fabien \textsc{Delhomme} \thanks{sous les précieux conseils de Vilmos \textsc{Komornik}}}

\begin{document}

\newcommand{\Z}[1]{\mathbb{Z}/#1\mathbb{Z}}
\newcommand{\Ze}{\mathbb{Z}}
\newcommand{\N}{\mathrm{N}}
\newcommand{\C}{\mathbb{C}}
\newcommand{\R}{\mathbb{R}}
\newcommand{\Lun}{\mathrm{L}^1}
\newcommand{\Ker}{\mathrm{Ker}}
\newcommand{\e}{\textrm{e}}
\newcommand{\inv}[1]{#1^\ast}
\newcommand{\norminf}[1]{\| #1 \|_{\infty}}
\newcommand{\Calg}{$\mathcal{C}^\ast$-algèbre }
\newcommand{\Calgs}{$\mathcal{C}^\ast$-algèbres }

\newtheorem{myth}{Theorème}
\newtheorem{mydef}{Définition}
\newtheorem{cor}{Corollaire}
\newtheorem{ex}{Exemple}

\maketitle

\AtBeginSection[]{
  \begin{frame}{Sommaire}
  \small \tableofcontents[currentsection, hideothersubsections]
  \end{frame} 
}
\section{Introduction et premiers exemples}

%On cherche à motiver l'étude des algèbres de banach, en commençant par les définir, puis en finissant sur le lemme de weiner, qui
%permet de motiver l'étude de ce genre d'algèbre puisque certaines preuves deviennent ainsi très élégantes



\subsection{Contexte}

\begin{frame}{Qu'est-ce qu'une algèbre de Banach ?}
    \begin{mydef}
        Une algèbre de Banach $A$ est un espace vectoriel complexe normé \emph{complet}, tel que :
        \begin{align*}
            x(yz) &= (xy)z\\
            (x + y)z &= xz + yz\\
            x(y+z) &= xy + yz \\
            \alpha(xy) &= (\alpha x)y = x(\alpha y)
        \end{align*}
       
        Et munit d'une norme d'algèbre :
        \begin{displaymath}
            \forall x,y \in A \quad \| x y \| \leq \|x\| \|y\|
        \end{displaymath}
    \end{mydef}
\end{frame}

\begin{frame}{Exemples}
    
\begin{ex}
    Quelques exemples d'algèbres de Banach  :

    \begin{itemize}
        \item L'espace des matrices complexes muni de la multiplication usuelle et d'une norme subordonnée. 
        \item L'ensemble des fonctions $\Lun$ munies de la convolution. On voit en particulier qu'on ne demande pas
            l'existence d'un élément neutre pour la loi multiplicative.
    \end{itemize}
    \end{ex}
\end{frame}



\begin{frame}{Ajout d'une unité}
    On peut toujours ajouter une unité dans une algèbre de Banach. Soit $A$ une telle Algèbre.
    On pose :
    $\hat{A} = \{ ( x, \alpha) \ | \ x \in A \, ,\, \alpha \in \C \}$. On définit :
    \[
        (x, \alpha) ( y, \beta) = (xy + \alpha y + \beta x, \alpha \beta )
        \]
    et 
    \[
        \| (x, \alpha) \| = \|x\| + |\alpha|
        \]
    Finalement, on pose $e = (0, 1)$
    \begin{myth}
        $\hat{A}$ est une algèbre de Banach munit d'un élément neutre, et isomorphe à $A$ par l'application $x \to (x, 0) $.
    \end{myth}
\end{frame}

\begin{frame}{Spectres, et premières propriétés}
    \begin{mydef}
        Le \emph{spectre} d'un élément $x \in A$ est défini par :
        \[
            \sigma(x) := \{ \lambda \in \C \ | \ \lambda e - x  \ \text{n'est pas inversible} \} 
        \]
    \end{mydef}
    On définit aussi $\rho(x) := \sup \{ |\lambda| \, , \lambda \in \sigma(x) \}$

    On a les premières propriétés :

    \begin{myth}
        \begin{itemize}
                \item Le spectre est compact et non vide.
                \item $\rho(x) = \lim_{n \longrightarrow \infty} \| x^n \|^\frac{1}{n} = \inf_{n \geq 1} \| x^n \|^\frac{1}{n}$
        \end{itemize}
        Donc en particulier $\rho(x) \leq \| x \| $.
    \end{myth}
\end{frame}

%TODO: les trucs en dessous sont en vrac !
\begin{frame}{Théorème de \textsc{Gelfand}-\textsc{Mazur}}
    Que peut-on dire d'une algèbre où tous les éléments non nulles sont inversibles ?

    \begin{myth}[\textsc{Gelfand}-\textsc{Mazur}]
        Si $A$ est une algèbre de Banach, telle que tous les éléments non nulles sont inversibles, alors $A$ est isomorphe à
        $\C$. De plus, l'isomorphisme est une isométrie.  
    \end{myth}
\end{frame}

\begin{frame}{Les idéaux}
    \begin{mydef}[Idéal]
        Soit $J \subset A$ un sous espace vectoriel de A. $J$ est un idéal de $A$ si 
        \[
            \forall x, y \in A*J \quad x*y \in J
        \] 
        Si $J \not = A$ et $J \not = \{0\}$ alors $J$ est un \emph{idéal propre} de $A$. Les \emph{idéaux maximaux} sont des idéaux propres qui ne sont
        contenus dans aucun autre idéal propre.
    \end{mydef}
\end{frame}

\begin{frame}{Merveilleux théorème}
    Voici l'outil fondamental de la théorie des Algèbres de Banach commutative.

       \begin{myth}[Caractérisation des inversibles]
           Notons $\Delta$ l'ensemble des homomorphismes de $A$. Alors :
           \begin{itemize}[<+->]
               \item Tout idéal maximal est le noyau d'un homomorphismes de $\Delta$, et \emph{vice versa}.
               \item $x \in A$ est inversible si et seulement si $h(x) \not = 0$ pour tout $h \in \Delta$
               \item $x \in A$ est inversible si et seulement si $x$ n'appartient à aucun idéal propre de $A$.
               \item $\lambda \in \sigma(x) $  si et seulement si $h(x) = \lambda $ pour un $h \in \Delta$
           \end{itemize}

           L'ensemble $\Delta$ gardera cette signification jusqu'à la fin du document.
       \end{myth} 
\end{frame}

\begin{frame}{Lemme de Wiener - Énoncé}
    \begin{myth}
        Soit $f$ une fonction de $\R^n$, et soit $(a_m)_{m \in \Ze} \in \R^{\Ze} $ telle que 
        \[
            f(x) = \sum_{m\in \Ze } a_m \textrm{e}^{im*x} \quad \text{,} \quad \sum_{m \in \Ze } | a_m | < \infty 
        \]

        Si $f$ ne s'annule jamais sur $\R^n$, alors il existe $(b_m)_{m \in \Ze} $ telle que :
        \[
            \frac{1}{f(x)} = \sum_{m \in \Ze} b_m \e^{im*x} \quad \text{,} \quad \sum_{m \in \Ze } | b_m | < \infty 
        \]
    \end{myth}
\end{frame}

\begin{frame}{Lemme de Wiener - Preuve}

\end{frame}

\begin{frame}{La transformée de Gelfand}
    On définit la transformée de Gelfand par :
    \begin{mydef}
        \[
            \hat{x} : h \in \Delta \longrightarrow h(x) \in \C 
        \]
        Et on note $\hat{A} = \{ \hat{x} \ | \ x \in A \}$
    \end{mydef}
\end{frame}

\begin{frame}
    \begin{myth}{Propriétés de la transformée de \textsc{Gelfand}}
        \begin{itemize}
            \item La transformée de \textsc{Gelfand} est un homomorphisme de $A$ vers $\hat{A}$, de noyau le radical de $A$.
            \item La transformée de \textsc{Gelfand} est donc un isomorphisme si $A$ est semi simple !
                \item On a aussi la comparaison :
                    \[
                        \| \hat{x} \|_{\infty} = \rho(x) \leq \| x \|
                    \]
                \item Et $\rho(x) = 0 \iff x \in \mathrm{rad}(A) $
        \end{itemize}
        \end{myth}

        Avec :
        \begin{mydef}
            \[
                \norminf{\hat{x}} = \sup_{h \in \Delta } | \hat{x}(h) |
            \]
        \end{mydef}
\end{frame}


\begin{frame}{Involution}
    \begin{mydef}[Involution]
        Une involution est une fonction $x \in A \longrightarrow \inv{x} \in A$ qui a les propriétés suivantes :
        \begin{align*}
            \inv{ (x + y) }  &= \inv{x} + \inv{y}\\
            \inv{( \lambda x) } &= \overline{\lambda} * \inv{x}\\
            \inv{( x * y) } &= \inv{ y } * \inv{x}\\
            (x^\ast)^\ast &= x 
        \end{align*}
        Et l'algèbre $A$ n'a pas besoin d'être commutative dans cette définition.

        On dit que $x \in A $ est \emph{hermitien} si et seulement si $\inv{x} = x $.
    \end{mydef}
\end{frame}

\begin{frame}{Les \Calgs}
    \begin{mydef}[Les \Calgs]
        Une \Calg est une algèbre de Banach munie d'une involution telle que : 
        \[
            \| \inv{x} * x \| = \| x \|^2 \quad \forall x \in A
        \]
        Cela implique aussi que :
        \[
            \| \inv{x} \| = \| x \|
        \]
        Et ainsi :
        \[
            \| x \inv{x} \| = \| x \| \| \inv{x} \|
        \]
    \end{mydef}
\end{frame}

\begin{frame}{Théorème de \textsc{Gelfand}-\textsc{Naimark}}
    \begin{myth}[Théorème de \textsc{Gelfand}-\textsc{Naimark}]
        Si $A$ est une \Calg \emph{commutative}, et en notant $\Delta$ un idéal maximum , alors la transformée de Gelfand est une isométrie de $A$ vers $\mathcal{C}(\Delta)$
        avec la propriété :
        \[
            h(\inv{x}) = \overline{h(x)}
        \]
        En particulier : $x$ est hermitien si et seulement si $\hat{x}$ est une fonction réelle.
    \end{myth}
\end{frame}

%Références================================================================================

\begin{frame}
\bibliographystyle{plain}
\bibliography{../bibliographie/bibliographie.bib}
\end{frame}
\end{document}
