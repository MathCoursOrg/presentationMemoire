\documentclass[12pt]{beamer}
\usepackage[utf8]{inputenc}
\usepackage[T1]{fontenc}
\usepackage[francais]{babel}
\usepackage{lmodern}
\usepackage{amsmath, amssymb, graphicx, amsthm, hyperref}
%\usepackage{fancyhdr}
%\usepackage{pslatex} % À enlever si problème, soit disant pour rendre le document plus "lisible" 

\usetheme{Madrid}

\title{La transformée de \textsc{Gelfand}}
\subtitle{Lien avec la physique, et théorie des algèbres de Banach}
\author{Fabien \textsc{Delhomme} \thanks{sous les précieux conseils de Vilmos \textsc{Komornik}}}

\begin{document}

\newcommand{\Z}[1]{\mathbb{Z}/#1\mathbb{Z}}
\newcommand{\Ze}{\mathbb{Z}}
\newcommand{\N}{\mathrm{N}}
\newcommand{\C}{\mathbb{C}}
\newcommand{\R}{\mathbb{R}}
\newcommand{\Lun}{\mathrm{L}^1}
\newcommand{\Ker}{\mathrm{Ker}}
\newcommand{\e}{\textrm{e}}
\newcommand{\inv}[1]{#1^\ast}

\newtheorem{myth}{Theorème}
\newtheorem{mydef}{Définition}
\newtheorem{cor}{Corollaire}
\newtheorem{ex}{Exemple}

\maketitle

\AtBeginSection[]{
  \begin{frame}{Sommaire}
  \small \tableofcontents[currentsection, hideothersubsections]
  \end{frame} 
}
\section{Introduction et premiers exemples}

%On cherche à motiver l'étude des algèbres de banach, en commençant par les définir, puis en finissant sur le lemme de weiner, qui
%permet de motiver l'étude de ce genre d'algèbre puisque certaines preuves deviennent ainsi très élégantes



\subsection{Contexte}

\begin{frame}{Qu'est-ce qu'une algèbre de Banach ?}
    \begin{mydef}
        Une algèbre de Banach $A$ est un espace vectoriel complexe normé \emph{complet}, tel que :
        \begin{align*}
            x(yz) &= (xy)z\\
            (x + y)z &= xz + yz\\
            x(y+z) &= xy + yz \\
            \alpha(xy) &= (\alpha x)y = x(\alpha y)
        \end{align*}
       
        Et munit d'une norme d'algèbre :
        \begin{displaymath}
            \forall x,y \in A \quad \| x y \| \leq \|x\| \|y\|
        \end{displaymath}
    \end{mydef}
\end{frame}

\begin{frame}{Ajout d'une unité}
    On peut toujours ajouter une unité dans une algèbre de Banach. Soit $A$ une telle Algèbre.
    On pose :
    $\hat{A} = \{ ( x, \alpha) \ | \ x \in A \, ,\, \alpha \in \C \}$. On définit :
    \[
        (x, \alpha) ( y, \beta) = (xy + \alpha y + \beta x, \alpha \beta )
        \]
    et 
    \[
        \| (x, \alpha) \| = \|x\| + |\alpha|
        \]
    Finalement, on pose $e = (0, 1)$
    \begin{myth}
        $\hat{A}$ est une algèbre de Banach munit d'un élément neutre, et isomorphe à $A$ par l'application $x \to (x, 0) $.
    \end{myth}
\end{frame}

\begin{frame}{Spectres}

\end{frame}

\subsection{Premiers théorèmes et applications}


\begin{frame}{Résultats sur les spectres}
\end{frame}
\begin{frame}{Caractérisation des éléments inversibles}
\end{frame}
\begin{frame}{Premiers exemples}
%pour donner de l'appétit aux matheux

\end{frame}

\subsubsection{Le lemme de \textsc{Wiener}}

\begin{frame}{Lemme de Wiener - Énoncé}
    \begin{myth}
        Soit $f$ une fonction de $\R^n$, et soit $(a_m)_{m \in \Ze} \in \R^{\Ze} $ telle que 
        \[
            f(x) = \sum_{m\in \Ze } a_m \textrm{e}^{im*x} \quad \text{,} \quad \sum_{m \in \Ze } | a_m | < \infty 
        \]

        Si $f$ ne s'annule jamais sur $\R^n$, alors il existe $(b_m)_{m \in \Ze} $ telle que :
        \[
            \frac{1}{f(x)} = \sum_{m \in \Ze} b_m \e^{im*x} \quad \text{,} \quad \sum_{m \in \Ze } | b_m | < \infty 
        \]
    \end{myth}
\end{frame}

\begin{frame}{Lemme de Wiener - Preuve}

\end{frame}

\subsection{Lien avec la physique quantique}

\subsection{Construction de }

\section{Conclusion}

\begin{frame}
\bibliographystyle{plain}
\bibliography{../bibliographie/bibliographie.bib}
\end{frame}
\end{document}
